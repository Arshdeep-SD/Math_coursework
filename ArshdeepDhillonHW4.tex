\documentclass[11pt,letter paper]{report}
\usepackage{amssymb,amsfonts,amsmath,color,graphicx}
\usepackage{tikz} 

\voffset=-3cm
\hoffset=-2.25cm
\textheight=24cm
\textwidth=17.25cm

\begin{document}
\noindent{\em Arshdeep Dhillon \hfill{September 4, 2019}}

\begin{center}
{\bf \Large Homework 4} 
\vspace{0.2cm}
\hrule
\end{center}

%
\section*{Problem Set 7}

%%
\subsection*{Problem 1}
If $E[X]=1$ and $Var(X)=5$, find
\begin{itemize}
\item[a.] $E[(2+X)^2]$
\item[b.] $Var(4+3X)$
\end{itemize}
{\bf \underline{Solution}}: 
\begin{itemize}
\item[a.] $E[(2+X)^2]=E[(4+4X+X^2]=E[4+4X]+E[X^2]$, $E[aX+b]=aE[X]+b$ \it{from lecture 16}\\
$E[(2+X)^2]=4+4E[X]+Var[X]=4+4+5=\underline{\underline{13}}$
\item[b.] $Var(4+3X)=3^2Var(X)=\underline{\underline{45}}$, $Var[aX+b]=a^2Var(X)$ \it{from lecture 16}
\end{itemize}

%%
\subsection*{Problem 6}
Let $X$ be a binomial random variable with parameters $n$ and $p$. Show that $$E[\frac{1}{1+X}]=\frac{1-(1-p)^{n+1}}{(n+1)p}$$
{\bf \underline{Solution}}:\\
By the definition of expectation of a random variable $X$,
\begin{flalign*}
E[\frac{1}{1+X}] &=\sum_{k=0}^n\frac{1}{k+1}\binom{n}{k}p^k(1-p)^{n-k} \text{  as, } p_X(i)=\binom{n}{i}p^i(1-p)^{n-i} &\\
&=\sum_{k=0}^n\frac{n!}{(n-k)!(k+1)!)}p^k(1-p)^{n-k} &\\
&=\frac{1}{(n+1)p}\sum_{k=0}^n\frac{(n+1)!}{(n-k)!(k+1)!}p^{k+1}(1-p)^{n-k}=\frac{1}{(n+1)p}\sum_{k=0}^n\binom{n+1}{k}p^{k+1}(1-p)^{n-k} &\\
&=\frac{1}{(n+1)p}[\sum_{k=0}^{n+1}\binom{n+1}{k}p^{k+1}(1-p)^{n+1-k}-(1-p)^{n+1}] &\\
&\text{by Binomial theorem }\sum_{k=0}^n=\binom{n}{i}p^i(1-p)^{n-i}=1 &\\
&\therefore E[\frac{1}{1+X}]=\frac{1-(1-p)^{n+1}}{(n+1)p} &
\end{flalign*}

%%
\subsection*{Problem 9}
Let $X$ be a Poisson random variable with parameter $\lambda$. What value of $\lambda$ maximizes $(P(X=k)$, $k\ge 0$?\\[0.1cm]
{\bf \underline{Solution}}:
$$P((X=k))=p_X(k)=\frac{\lambda^k}{k!}e^{-\lambda}\, k\ge 0k$$
Writing $p_X(k)$ as a function of lambada, 
$$f(\lambda)=\frac{\lambda^k}{k!}e^{-\lambda}\, \exists k\in \mathbb{N}, k\ge0$$
The critical points of $f(\lambda)$ are the solutions to $f'(\lambda)=0$. 
$$f'(\lambda)=\frac{1}{k!}[k\lambda^{k-1}e^{-\lambda}-\lambda^ke^{-\lambda}] \text{  by Chain rule}$$
$$\therefore f'(\lambda)=0\to \frac{1}{k!}[k\lambda^{k-1}e^{-\lambda}-\lambda^ke^{-\lambda}]=0$$
$$\implies \lambda=k$$
To check whether a given critical point $X$ is a maxima $f''(X)<0$,
$$f''(\lambda)=\frac{1}{k!}(k[(k-1)\lambda^{k-2}e^{-\lambda}-\lambda^{k-1}e^{-\lambda}]-f'(\lambda))$$
at $\lambda=k$
$$f''(k)=\frac{e^{-k}}{k!}[(k-1)k^{k-1}-k^{k}]\implies f''(k)<0 \text{  as,}(k-1)k^{k-1}<k^k$$
$$\therefore f''(k)<0 $$
Therefore, $(P(X=k)$ is maximum when $\lambda=\underline{\underline{k}}$.

%%
\subsection*{Problem 10}
Show that $X$ is a Poisson random variable with parameter $\lambda$, then $$E[X^n]=\lambda E[(X+1)^{n-1}]$$
Compute $E[X^3]$.\\[0.1cm]
{\bf \underline{Solution}}:\\
By the definition of expectation of a random variable $X$,
\begin{flalign*}
E[X^n] &=\sum_{k\in \mathbb{N}}k^n\frac{\lambda^k}{k!}e^{-\lambda} \text{  as, } p_X(i)=\frac{\lambda^i}{i!}e^{-\lambda} &\\
&=\lambda\sum_{k\in \mathbb{N}}k^{n-1}\frac{\lambda^{k-1}}{(k-1)!}e^{-\lambda} &\\
&\sum_{k\in \mathbb{N}}k^{n-1}\frac{\lambda^{k-1}}{(k-1)!}e^{-\lambda} \text{ represents the expected value of the random variable }(X+1)^{n-1} &\\
&\therefore E[X^n]=\lambda E[(X+1)^{n-1}] &
\end{flalign*}
Using this result, 
\begin{flalign*}
E[X^3] &=\lambda E[(X+1)^2] &\\
&=\lambda E[X^2+2X+1] &\\
&=\lambda(E[X^2]+2E[X]+1) \text{  as, } E[1]=1 &\\
&=\lambda(E[X^2]+2\lambda+1) \text{  as, } E[X]=\lambda \text{ (Poisson Distribution)} &\\
&=\lambda(\lambda(\lambda+1)+2\lambda+1) \text{  as, } E[X^2]=\lambda E[X+1] \text{ (Poisson Distribution)} &\\
&=\lambda^3+3\lambda^2+\lambda &\\
\end{flalign*}
$$\therefore E[X^3]=\underline{\underline{\lambda^3+3\lambda^2+\lambda}}$$

%
\section*{Problem Set 8}

%%
\subsection*{Problem 1}
The lifetime in hours of an electronic tube is a random variable having a probability density function given by 
$$f(x)=\begin{cases}
0,& x<0\\
xe^{-x},& x\geq 0
\end{cases}$$
Compute the expected lifetime of such a tube.\\[0.1cm]
{\bf \underline{Solution}}:\\
Let $X$ be a random variable that denotes the number of number of hours that the electronic tube is alive. The probability that a randomly selected tube will be functional for $x_0$ hours is given by the probability density function $f(x)$. Therefore, 
$$p_X(x_0)=f(x_0)=x_0e^{-x_0}$$
Using the definition of expected value of a random variable, 
$$E[X]=\int_{-\infty}^{\infty}xp_X(x)dx=\int_{-\infty}^{\infty}x^2e^{-x}dx$$
$$\implies E[X]=\int_0^{\infty}x^2e^{-x}dx=2$$
Therefore, the expected lifetime of one of these electronic tubes chosen randomly is \underline{\underline{$2$}} hours.

%%
\subsection*{Problem 2}
A point is chosen at random on a line segment of length $L$. Interpret this statement, and find the probability that the ratio of the shorter to the longer segment is less than $\frac{1}{4}$.\\[0.1cm]
{\bf \underline{Solution}}:\\
When a point on the line segment is chosen, it can be thought that it is divided into two segments of lengths $x\le L$ and $(1-x)L$. Random variable $X$ can be defined as length of the segment that is created when a point is chosen at random and one end of the original line segment. As all points are equally likely to be selected $X$ is uniformly distributed in $[0,L]$, as the chosen point must lie on the line segment. 
As the ratio of the shorter to the longer segment must be found the fraction must be flipped when the ration becomes greater than $1$, as then the other part of the segment is shorter. The ratio of the two lengths is $1$ is when the point chosen bisects the line segment and the value of $X$ at this point is $\frac{L}{2}$. 
The ratio of the two lengths can be represented as follows,
$$r(X)=\begin{cases}
\dfrac{X}{1-X},& 0\le X\le \frac{L}{2}\\
\dfrac{1-X}{X},& L\ge X\ge \frac{L}{2}\\
\end{cases}$$
The cases when the ratio of the shorter length to the longer length is less than $\frac{1}{4}$ are, 
$$r(X)<\frac{1}{4}\implies X\in [0,\frac{L}{5}) \cup (\frac{4L}{5}, L]$$
Therefore the probability that $r(X)<\frac{1}{4}$ is given by, 
$$P(0\le X<\frac{L}{5})+P(\frac{4L}{5}<X\le L)=\int_0^{\frac{L}{4}}\frac{1}{L-0}dx+\int_{\frac{4L}{5}}^L\frac{1}{L-0}dx=2\frac{1}{5}$$
$$\therefore P(X\in [0,\frac{L}{5}) \cup (\frac{4L}{5}, L)=\underline{\underline{\frac{2}{5}}}$$

%%
\subsection*{Problem 9}
Show that, if $Z$ is a standard normal random variable, then, for $x>0$
\begin{itemize}
\item[a.] $P(Z>x)=P(Z<-x)$
\item[b.] $P(\lvert Z\rvert>x)=2P(Z>x)$
\item[c.] $P(\lvert Z\rvert<x)=2P(Z<x)-1$
\end{itemize}
{\bf \underline{Solution}}:\\
If a normal distribution is said to be 'standard' then it is known that (mean)$\mu=0$ and (variance)$\sigma^2=1$. Therefore the probability density function can be written as follows, 
$$p_Z(z)=\dfrac{e^{\frac{-z^2}{2}}}{\sqrt{2\pi}}\implies P(Z\le x)=\int_{-\infty}^xp_Z(z)dz=P(Z\ge -x)=\int_{-x}^{\infty}p_Z(z)dz$$
\begin{itemize}
\item[a.] $P(Z>x)=1-P(Z\le x)$\\
$\implies P(Z>x)=1-P(Z\ge -x)$\\
$\implies P(Z>x)=(Z<-x)$, hence proven.
\item[b.] $P(\lvert Z\rvert>x)\to P(Z>x)$ or $P(Z<-x)$\\
$\implies P(\lvert Z\rvert>x)=P(Z>x)+P(Z<-x)=2P(Z>x)$, as shown in 'a.'\\
$\therefore P(\lvert Z\rvert>x)=2P(Z>x)$, hence proven.
\item[c.] $P(\lvert Z\rvert<x)\to P(Z<x)$ and $P(Z>-x)$\\
$\implies P(\lvert Z\rvert>x)=P(Z>x>-Z)=1-P(\lvert Z\rvert>x)=2P(Z>x)$\\
$\implies P(\lvert Z\rvert>x)=1-2P(Z>x)=1-2(1-P(Z<x))$\\
$\therefore P(\lvert Z\rvert>x)=2P(Z<x)$, hence proven.
\end{itemize}

%%
\subsection*{Problem 10}
The median of a continuous random variable having distribution function $F$ is that value $m$ such that $F(m)=1/2$. That is, a random variable is just as likely to be larger than its median as it is to be smaller. Find the median of $X$ if $X$ is
\begin{itemize}
\item[a.] uniformly distributed over an interval $(a,b)$.
\item[b.] normal with parameters $\mu$ and $\sigma^2$.
\item[c.] exponential with rate $\lambda$.
\end{itemize}
{\bf \underline{Solution}}:
\begin{itemize}
\item[a.] The distribution function of a uniformly distributed random variable for $x\in (a,b)$ is $F(x)=\frac{x}{b-a}$. At the median $m$, the value of this function is $\frac{1}{2}$. Therefore, 
$$F(m)=\frac{m}{b-a}=\frac{1}{2}\implies m=\underline{\underline{\frac{b-a}{2}}}$$
\item[b.] The normal distribution function is symmetric around the point $x=\mu$, therefore, $P(X<\mu)=P(X>\mu)$(also from Problem 9 'a.'), and $P(X<\mu)+P(>\mu)=1$. 
$$\therefore P(X<\mu)=\frac{1}{2}\implies \text{median }m=\underline{\underline{\mu}}$$  
\item[c.] The distribution function of a exponentially distributed random variable for $x\ge 0$ is $F(x)=\int_0^x\lambda e^{-\lambda x}dx=1-e^{-\lambda x}$. At the median $m$, the value of this function is $\frac{1}{2}$. Therefore, 
$$F(m)=1-e^{-\lambda m}=\frac{1}{2}\implies m=\underline{\underline{\frac{\ln{2}}{\lambda}}}$$
\end{itemize}

\end{document}